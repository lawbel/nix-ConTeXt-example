% For help learning ConTeXt, or as a reference material, a good resource is
% https://wiki.contextgarden.net/

\environment[style]
\environment[images]

\starttext

\placebookmarks  % generate PDF bookmarks
  [chapter, section]  % show these headings in the bookmarks
  [chapter, section]  % when the file is first opened, expand these headings

\placecontent  % place a default table of contents

\chapter[title={Chapter One}]

\section[title={Code Samples}]

Below is a basic setup for a \type{flake.nix} file to build
a \CONTEXT{} document.

\starttyping
stdenv.mkDerivation {
    buildInputs = [texlive.combined.scheme-context];
    buildPhase  = "FONTCONFIG_FILE=${fonts} context main.tex";
    unpackPhase = "mkdir -p $out && cp main.pdf $out"
}
\stoptyping

\section[title={Graphics & Mathematics}]

To show off some graphics and maths typesetting we present the Residue
Theorem, a~mathematical result from the field of Complex Analysis:

\blank

\definemathcommand[wind] {\mfunction{I}}
\definemathcommand[Res]  {\mfunction{Res}}

\startcolumns[n=2]
  Suppose that \m{U \subset \complexes} is open and simply||connected,
  containing a set of points \m{\fenced[brace]{a_1, \ldots, a_n} = A}. Let
  \m{f} be holomorphic on \m{U_0 = U \setminus A} and \m{γ \subset U_0} be
  some closed rectifiable curve. See \in{figure}[fig:residue] for an
  illustration. Also, denote the winding number of \m{γ} around each point
  \m{a \in A} by \m{\wind(γ, a)}.

  Then the integral of \m{f} around \m{γ} equals \m{2πi} the sum of residues
  of \m{f} over \m{A}, each counted as many times as \m{γ} winds around
  that point:
\column
  \startplacefigure
    [location=text,
     reference=fig:residue,
     title={The setting in \m{\complexes}}]
    \useMPgraphic{residue}
  \stopplacefigure
\stopcolumns

\startformula
  \frac{1}{2πi} \, \oint_γ f(z) \, dz
    = \sum_{a \in A} \wind(γ, a) \, \Res(f, a)
\stopformula

\stoptext
