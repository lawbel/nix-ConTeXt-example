\startenvironment[fonts]

% we wrap this section in a setup for ease, so that we don't have to worry
% about trailing whitespace creeping in when we switch to these fonts
% (particularly in-line)
\startsetups[font:cascadia]

% Here we set up 'Cascadia Code' font, a monospace font.
%
% It can be difficult to set up a font that doesn't come bundled
% with ConTeXt. To help with that, the below links have some good advice:
%
% (1) https://wiki.contextgarden.net/Fonts
% (2) https://wiki.contextgarden.net/Use_the_fonts_you_want
%
% Step 1/3: choose some convenient names to associate with the raw font files.
% The raw font files can be found by examining the output of
%
%   nix run .#query-fonts -- --list --file cascadia
%
% which should look something like (omitting some of the output, as it
% includes a lot of different information):
%
%   familyname    weight  style   ...  filename
%
%   ...
%   cascadiacode  normal  normal  ...  /nix/store/.../CascadiaCode-Regular.otf
%   cascadiacode  normal  italic  ...  /nix/store/.../CascadiaCode-Italic.otf
%   cascadiacode  bold    normal  ...  /nix/store/.../CascadiaCode-Bold.otf
%
% So, for the bold version of cascadia, we can use file:CascadiaCode-Bold
% as that is the base file-name from the last line here - drop the
% extenstion `.otf` and don't include the path to the file.
%
% Following this process for all variants (but opting to use semi-light
% versions), we get this:
\starttypescript[cascadia]
  \definefontsynonym[Cascadia-Regular]    [file:CascadiaCode-SemiLight]
  \definefontsynonym[Cascadia-Italic]     [file:CascadiaCode-SemiLightItalic]
  \definefontsynonym[Cascadia-Bold]       [file:CascadiaCode-SemiBold]
  \definefontsynonym[Cascadia-BoldItalic] [file:CascadiaCode-SemiBoldItalic]
\stoptypescript

% step 2/3: map 'ConTeXt basics names' (for monospace fonts these include
% "Mono", "MonoItalic", "MonoBold", and "MonoBoldItalic") to the names above.
%
% I don't know a good way to find these 'ConTeXt basics names' other than by
% looking at existing font definitions that come with ConTeXt. For reference,
% here are the names for different font types (which can also be found at
% `tex/context/base/mkiv/font-sel.lua`, search for "m_synonym", in your
% ConTeXt distribution)
%
% - rm ('roman' / serif):
%   - Serif            (tf)
%   - SerifBold        (bf)
%   - SerifItalic      (it)
%   - SerifSlanted     (sl)
%   - SerifBoldItalic  (bi)
%   - SerifBoldSlanted (bs)
%   - SerifCaps        (sc)
% - ss (sans-serif):
%   - Sans             (tf)
%   - SansBold         (bf)
%   - SansItalic       (it)
%   - SansSlanted      (sl)
%   - SansBoldItalic   (bi)
%   - SansBoldSlanted  (bs)
%   - SansCaps         (sc)
% - tt (teletype / monospace):
%   - Mono             (tf)
%   - MonoBold         (bf)
%   - MonoItalic       (it)
%   - MonoSlanted      (sl)
%   - MonoBoldItalic   (bi)
%   - MonoBoldSlanted  (bs)
%   - MonoCaps         (sc)
% - mm (maths):
%   - MathRoman        (tf)
%   - MathBold         (bf)
% - hw (handwriting):
%   - Handwriting      (tf)
% - cg (calligraphy):
%   - Calligraphy      (tf)
%
% So if you are defining (say) a sans-serif font instead of a monospace one,
% you will want to use the names "Sans", "SansBold", "SansItalic", etc.
\starttypescript[cascadia]
  % add a fallback in case we try to use some characters that are missing
  % from the font
  \setups[font:fallback:mono]
  \definefontsynonym[Mono]           [Cascadia-Regular]    [features=default]
  \definefontsynonym[MonoItalic]     [Cascadia-Italic]     [features=default]
  \definefontsynonym[MonoBold]       [Cascadia-Bold]       [features=default]
  \definefontsynonym[MonoBoldItalic] [Cascadia-BoldItalic] [features=default]
\stoptypescript

% step 3/3: now define a 'typeface' from these fonts and give it a name
\starttypescript[cascadia]
  % note:
  %   tt   = teletype
  %   mono = monospace
  \definetypeface[cascadia] [tt] [mono] [cascadia] [default]
\stoptypescript

\stopsetups


\setup[font:cascadia]  % run the above setup

\stopenvironment
